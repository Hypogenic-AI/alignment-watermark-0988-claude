\section{Conclusion}
\label{sec:conclusion}

We have presented empirical evidence that alignment training creates an implicit watermark in LLM outputs.
Across three model families, three detection paradigms, and 12 independent comparisons, aligned models are consistently and significantly more detectable than their base counterparts (sign test $p = 0.006$).
The watermark manifests as increased structural regularity---reduced sentence-length variability and higher token-level predictability---and is detectable by both statistical methods (AUROC +0.09 to +0.35) and by other LLMs (TPR 97.5--100\% on aligned text vs.\ 85--96.3\% on base text).

These findings have practical implications.
If alignment is the watermark, then explicit watermarking may be unnecessary for aligned models, which already carry a durable detection signal.
Detection research should focus on exploiting the alignment watermark rather than imposing external marks.
Policymakers should recognize that the very process of making AI helpful provides a natural accountability mechanism.

The fundamental insight is this: there is an inherent tension between making AI useful and making AI undetectable.
A perfect base model approximates the human distribution and approaches undetectability.
The moment we align that model to serve human preferences, we create a distributional shift that guarantees detectability.
Alignment is not merely a target for detection---it \emph{is} the detection signal.

\para{Future work.}
Three directions are most promising.
First, testing across additional domains (creative writing, conversation, code) and model families would establish the generality of the alignment watermark.
Second, isolating the contributions of different alignment stages (SFT vs.\ RLHF vs.\ DPO) on the same base model would clarify which stage creates the strongest signal.
Third, adversarial robustness testing---whether paraphrasing, back-translation, or prompt engineering can erase the alignment watermark while preserving helpfulness---would determine whether the watermark represents a fundamental trade-off or an erasable artifact.
